\documentclass[9pt]{beamer}

% Preamble

% Pakker
\usepackage[utf8]{inputenc}
\usepackage{tcolorbox}
\usepackage[T1]{fontenc}
\usepackage{babel}
\usepackage{amssymb}
\usepackage{blkarray}
\usepackage{tikz}
\usepackage{fourier}

% Udseende
\newlength{\sbwidth}
\setlength{\sbwidth}{1.6cm} % Bredden på sidebaren
\usetheme[width=\sbwidth]{Berkeley} % Layout er baseret på dette tema, men farvene er ændret
\usefonttheme{serif}
\usecolortheme{beaver}
\logo{\includegraphics[width=\sbwidth]{img/aau-logo.png}}
\setbeamertemplate{navigation symbols}{}
\setbeamertemplate{footline}[frame number]
\setbeamerfont{frametitle}{series=\bfseries}
\setbeamerfont{title in sidebar}{series=\bfseries}
\setbeamerfont{title}{size=\huge,series=\bfseries}

% Farver
\colorlet{AAUgrey}{gray!10}
\definecolor{AAUgrey1}{RGB}{ 84, 97,110}
\definecolor{AAUgrey2}{RGB}{104,119,132}
\definecolor{AAUgrey3}{RGB}{162,172,182}
\definecolor{AAUgrey4}{RGB}{222,223,226}
\definecolor{AAUblue1}{RGB}{ 33, 26, 82}
\definecolor{AAUblue2}{RGB}{ 89, 79,191}
\setbeamercolor*{palette primary}{fg=black,bg=white}
\setbeamercolor*{palette secondary}{bg=AAUgrey}
\setbeamercolor*{sidebar}{fg=black,bg=AAUgrey}
\setbeamercolor*{palette sidebar secondary}{fg=AAUgrey1}
\setbeamercolor*{section in sidebar shaded}{fg=AAUgrey3}
\setbeamercolor*{subsection in sidebar shaded}{fg=AAUgrey3}
\setbeamercolor*{title in sidebar}{fg=AAUblue1}
\setbeamercolor*{title}{fg=AAUblue1}
\setbeamercolor*{author in sidebar}{fg=black!80}
\setbeamercolor*{frametitle}{fg=AAUblue1,bg=white}
\setbeamercolor*{alerted text}{fg=AAUblue2}
\setbeamercolor*{item projected}{fg=white,bg=AAUblue1}
\setbeamercolor*{item}{fg=AAUblue1}

% Indholdsfortegnelse før hver nye sektion
\AtBeginSection[]
{
  \begin{frame}
    \frametitle{Indhold}
    \tableofcontents[currentsection]
  \end{frame}
}

% Align virker med pauser
\makeatletter
\let\save@measuring@true\measuring@true
\def\measuring@true{%
  \save@measuring@true
  \def\beamer@sortzero##1{\beamer@ifnextcharospec{\beamer@sortzeroread{##1}}{}}%
  \def\beamer@sortzeroread##1<##2>{}%
  \def\beamer@finalnospec{}%
}
\makeatother

% Definitioner, sætninger og eksempler udseende (brug thm selv til lemmaer og korollarer)
\newenvironment{exmp}[1]{\vspace{0.5em}\noindent\textbf{#1}\enspace}{\vspace{0.5em}}
\newenvironment{defn}[1]{\noindent\textbf{#1}\enspace}{}
\newenvironment{thm}[1]{\noindent\textbf{#1}\enspace\em}{}
\tcbuselibrary{breakable}
\tcbset{colback={AAUgrey}, arc=0mm, colframe=white,boxrule=0pt,toprule=8pt,bottomrule=0pt, breakable}
\newtcolorbox{theorembox}[2][]{coltitle=black,title={#2},fonttitle=\bfseries,#1, detach title,before upper={\tcbtitle\quad}}
\tcolorboxenvironment{defn}{bottomrule=8pt,sharp corners}
\tcolorboxenvironment{thm}{bottomrule=8pt,sharp corners}

% Diverse kommandoer
\newcommand{\N}{\mathbb{N}}
\newcommand{\R}{\mathbb{R}}
\newcommand{\vc}[1]{\mathbf{#1}}
% ...


% Information på forsiden
\title[P2 Eksamen]{Projekttitel}
\subtitle{Undertitel}
\author[Forfatter1, Forfatter2]{F.~Forfatter1 \and F.~Forfatter2}
\date[2020]{3. juni 2020}

\begin{document}

\frame{\titlepage}

% Her inkluderes indhold fra filer der ligger i mappen incl/
%\input{incl/frame1}...


% EKSEMPLER-----------------------------------
% Frame med eksempel på sætningsmiljøer
\section{Definitioner, sætninger og eksempler}
\begin{frame}{Eksempel på definition-, sætning- og eksempelmiljøer}
	Her er noget tekst
	\alert{highlighted}
	fordi det er vigtigt.

	\begin{defn}{Definition 1.1}
		Den første definition
	\end{defn}
	
	\begin{exmp}{Eksempel}
		Det første eksempel
	\end{exmp}

	\begin{thm}{Sætning 2}
		En sætning der er skrå.
		Her er noget matematik.
		\begin{align}
			\sum_{k=1}^n 2k-1 = n^2
		\end{align}
	\end{thm}
\end{frame}

% Eksempler på lister
\section{Lister}

% Frame med eksempel på punkter med effekter
\subsection{Punkter og effekter}
\begin{frame}{Et eksempel med brug af effekter på punkter}
	Her er der et frame med en punktopsætning.
	\begin{itemize} % Husk <#-> efter \item, hvor # er tallet på hvornår punktet dukker op
		\item<1-> Dette er første punkt.
		\item<2-> Dette er andet punkt på listen.
		\item<3-> Her er tredje punkt.
		\item<4-> Her er det fjerde punkt på listen.
	\end{itemize}
\end{frame}

% Eksempel på en nummereret liste
\subsection{Nummereret liste}
\begin{frame}{Eksempel på en nummereret liste}
	\begin{enumerate}
		\item Første skridt.
		\item Andet skridt.
		\item Tredje skridt.
	\end{enumerate}
\end{frame}

\end{document}
